\documentclass[conference]{IEEEtran}
\IEEEoverridecommandlockouts

%------------------------------------------------------------------------------

\usepackage{cite}
\usepackage{amsmath}
\usepackage{amsfonts}
\usepackage{amssymb}
\usepackage{graphicx}
\usepackage{url}
\usepackage{cite}
\usepackage{balance}
\usepackage{float}
\usepackage{threeparttable}

\usepackage{mathptmx}
\usepackage[scaled=.90]{helvet}
\usepackage{courier}

\usepackage{listings}
\usepackage{listings}
\lstset{
   language=C,
   basicstyle=\small,
   keywordstyle=\bfseries,
   identifierstyle=\ttfamily,
   stringstyle=\ttfamily,
   numbers=left,
   numberstyle=\tiny,
   stepnumber=1,
   numbersep=-5pt,
   showstringspaces=false
%   frame=single %trbl%
}

\usepackage[normalem]{ulem}

%------------------------------------------------------------------------------

\newcommand{\etal}{\emph{et al.}}
\newcommand{\eg}{\emph{e.g.}}
\newcommand{\ie}{\emph{i.e.}}
\newcommand{\etc}{\emph{etc.}}

%------------------------------------------------------------------------------

\begin{document}

%------------------------------------------------------------------------------

\title{A Distribuited Segment-based Approach for Topology-agnostic Nano Networks Routing} 

\author{
  \IEEEauthorblockN{a} \IEEEauthorblockA{Affiliation3\\ email3} \and
  \IEEEauthorblockN{b} \IEEEauthorblockA{Affiliation2\\ email2} }

\maketitle

%------------------------------------------------------------------------------

\begin{abstract}
In this paper we present a first effort for a distributed
segment-based approach to routing which exploits the already-proven
qualities of segment-based routing without requiring as graph topology
as input, and without the need for a centralized algorithm which
perform a setup phase.  Our contribution is not find the
best segment choice, but to show something that could really work in
such complex, irregular and large sized topologies.
\end{abstract}

%------------------------------------------------------------------------------

\IEEEkeywords{...}

%------------------------------------------------------------------------------

\section{Introduction}
Exploring long term alternatives to the CMOS technology is gaining
more and more revelance as the scaling trend of such devices continue
to introduce new challenges. Power density, defect tolerance, testing
costs and many other issues are only partially mitigated by multicore
approaches~\ref{TODO}, but growing computing requirements will eventually need
even more radical architectural modifications and new paradigms in
future computer design.

DNA-self assembled nanoscale architectures~\ref{TOOD} are emerging as promising
technology due its tera/peta scale of integration, defect tolerance
tolerance and potential computing capabilities~\ref{TODO}. This
technologies is certainly still at its early stage of development,
however different laboratory demos and prof-of-concepts architectures
have been presented~\ref{TODO}.

The main idea beyond this approach is to exploit the regularity and
stabiliy of DNA sequences in orders to create grids onto which
nano-devices (e.g. nanowires and CNFET~\ref{TODO}) can be attached by
designing approapriate DNA tags. A detailed description of the
chemical properties involved is beyond the scope of this paper, since
we focus of the main challenges that this new fabrication process
introduces in Computer Design. These are:
\begin{itemize}
\item small scale control of placement and connectivity
\item large scal randomness
\item high defect rates
\end{itemize}


This aspects of a DNA-self assembled process leads some architectural implications
when approaching to the Computer Design. Architecture must consist of
a distributed network of small computing and storage nodes, randomly
connected. Data will be travelling on packets that must be routed
using a topology-agnostic strategy, since no particular regularity can
be assumed in such network. 

Different works have been presented addressing the problem of topology
agnostic routing, such as~\ref{}. Althougth they show interesting
performances, the requirement for virtual channels resources makes
them not suitable for the particular DNA self assembled we are
focusing on. 

Other topology independent algorithms not using virtual channels are
based on turn proibition. Up-Down~\ref{} exploits the creation of a
spanning tree of the topology, then placing bidirectional restrictions
by avoiding a packet to traverse the same link in both up and down
directions. The hierarchical nature of this approach can lead to
uneven traffic distribution, with many packets traversing upper links
(near to the root). The~\ref{} mitigates this issue, making this
approach acceptable in small networks, however the whole restriction
placement is still fixed, depending on the particular root selected,
making these approaches certainly not scalable for the thousands nodes
scenario of DNA self assembled technology.

In~\ref{} authors present SR, a new approach the solves this limitation by
allowing turn restriction to be placed locally, independently from
other restriction. The whole network is partitioned in segments, and a
single bidirectional turn restrictions can be chosen for each
segment in order to guarantee deadlock freedom and connected networks.
This locality independence makes SR scalable for large-sized networks,
but its topology independency still requires the knowledge of the
topology graph in order to find the segments. The small scale control,
random placement and connectivity features discussed in the previous
section makes having such a topology graph impossible in DNA
self assembly, together with any other similar centralized algorithm using
graph description as input.

The contribution presented in this paper is a first effort for a
distribuited segment-based routing in DNA based networks (DiSR).  This
paper is not intended to show discover the "optimal" segment choice, ideally
reachable with a whole graph topology knowledge, but to demonstrate something could
really fit such complex, irregular and large sized topologies.

%------------------------------------------------------------------------------
\section{DiSR main elements}
%------------------------------------------------------------------------------

We distinct among two different kind of data stored at each node:
local environment data (LED) and dynamic behaviour status (DBS).

\subsection{Local Environment Data (LED)}

The LED is like a snapshot of the DiSR algorithm on each node,
consisting in the following variables:

segID:a value used to specify a segment id for the node. Depending on
the values of visited/tvisited variables below, the node can be
assigned to or candidate for a specific segment id.  

visited : a boolean value corresponding to the visited status. If visited is true
and segID is different from NULL, then segID specifies the segment id
the node belongs to. Note that a value different from NULL does not
necessarily mean the node has been assigned to some segment (e.g. when
searching starting segment is set visited by default even if not yet
assigned to any segment). When false, the node is not yet visited.

tvisited: a boolean value corresponding to the tvisited status. If
true, the node is being considered as candidate for a segment, and the
segID specifies the segment id for which the node it's candidate. When
false, the node is not currently tvisited.  

link\_visited[]: an array
of values representing information about attached links, that is, the
ID of the segment owning each link. When NULL, a link has not yet been
marked as visited. TDB: do we need to introduce some value for the
“bridge” status ? (e.g. a node connecting the terminal node of a
subnet with the starting one of another) 

link\_tvisited[]: an array of
values representing information about attached links, e.g. the ID of
the segment for which the link is candidate. When NULL, the link is
currently not tvisited.  starting: boolean, true if the node it's a
starting node for the subnet terminal:boolean, true if the node it's a
terminal node for the subnet.  subnet:the subnet the node belongs to.
If NULL, none.


\subsection{Dynamic Behaviour Status (DBS)}
In addition to the LED variables
described above, further information should be stored in order to
capture the current dynamic behaviour of the node. This is represented
by the dynamic behaviour status (DBS) variable, which strictly depends
on the LED data and the events occurring at the node. The DBS can have
the following values:

FREE: a node that has  not been yet considered  by the DiSR algorithm.
The node is not marked as visited/tvisited/terminal/starting.

BOOTSTRAP: a node which has been explicitly set as bootstrap node from
an upper layer via. This is required because a

STARTING\_SEGMENT\_REQUEST packet should be injected in order to start
the whole process.

ACTIVE\_SEARCHING: a node from which a find process of new segments has
been started and not yet cancelled or confirmed. This happens when all
the following conditions are matched: the node is marked as visited,
that is visited = TRUE 

CHECK: segID is set to some value reflecting
the segment request sent there is a link that is being investigated by
the node, that is a tvisited\_link marked with the same id of the node 

CANDIDATE: a node currently candidate for belonging to some segment
with id segid, not being itself the node from which the find process
was started. The following conditions should be matched: tvisited =
true, with segID=X different from node's id.  Two links set as
link\_tvisited with id X. Let’s refer them as  links 'i' and 'j'. Then
link\_tvisited[i] was set = X when the find process from an adjacent
node reached the current node, while link\_tvisited[j] has been set to
X when the node itself started to investigate its free links for the
segment request with id X. Note that while investigating its attached
links, if the find process fails along path 'j', the current node can
investigate another free node 'k' if suitable.  

CANDIDATE\_STARTING:
same as above, but the node is currently being considered as candidate
for starting segment. Two main differences: Can confirm
STARTING\_SEGMENT\_CONFIRM with the same segID when receiving a normal
SEGMENT\_REQUEST: Overwrites its status setting it to CANDIDATE for the
segID frees the links previously set as tvisited during flooding

ASSIGNED: a node for which the segment has been determined.  The
segment segID attribute value is set to some id X different from NULL.
The visited value is true NOTE: an ASSIGNED state is a quick temporary
state, since it becomes ACTIVE\_SEARCHING if a not (visited/tvisited)
link is suitable for searching new segments (see next\_visited
procedure on paper) 

PROCESSED: a node which should take no other
action, since all its links have been investigated. 

\subsection{Packets types Required}
TODO: a lot of description is already present elsewhere, make this
section smaller...

SEGMENT\_REQUEST: used to search candidates for a segment (not the
first of the subnet, managed separately as a starting segment).

SEGMENT\_CONFIRM: used to create a segment. When received, if the node
status is tvisited with a segID corresponding to the one indicated in
the packet, the node learns to belong to the segment associated to the
id. Further, it should forward this packet to the link where the
original SEGMENT\_REQUEST packet came from, so that all candidate nodes
can learn the segment id they belong to. The following LED shoud be
updated: The state of the node changes from tvisited to visited.  the
link\_tvisited previously set with the segID should be invalidated the
link\_visited variable associated to the incoming confirm should be set
as segID. Note that if the request was flooded across different links,
the link\_visited variable associated to the other paths shouldn’t be
set.  the link\_visited variable associated to the link from which the
request was originated should be set segID

STARTING\_SEGMENT\_REQUEST: the request for the first segment of a
subnet should be differentiated since all nodes (except the starting
one) must forward the packet using a flooding mechanism. This is
necessary since the request packet must reach the starting node, which
is the only node marked as visited. As stated elsewhere, it should
remembered that the first segment of each subnet it's  a loop that
beginning/ending on the starting node.


SEGMENT\_CANCEL: used when the process of searching a segment along a
certain path fails. When a node discovers that the path that includes
the node itself is no more suitable, a SEGMENT\_CANCEL is sent back
across the link from which the SEGMENT\_REQUEST was received.  Possible
reasons for that: The node it’s already candidate for another segment
The node it’s free but can’t find a free link (according the
cyclelinks timeout) The node it’s tvisited for the same request id,
previously forwarded that request along some free direction, but now
it’s receiving a a cancel request and has no more free links

\section{DiSR Execution Model}

\subsection{Creating the starting segment}

Injecting request: all nodes have status FREE, except for a
“bootstrap node” with status BOOTSTRAP, set by some signal from an
upper layer via. This bootstrap node sets itself as “visited” and
changes its status to ACTIVE\_SEARCHING, sending a
STARTING\_SEGMENT\_REQUEST across one of its not (visited/tvisited)
links and.  

Flooding: Each node receiving a STARTING\_SEGMENT\_REQUEST,
if not (visited/tvisited),  forwards  to the attached links that are
not already marked as visited/tvisited using a flooding mechanism.
Each of these links is then marked as tvisited with the segment id
associated to the packet. Note that a node that has already received a
STARTING\_SEGMENT\_REQUEST packet can simply ignore further packets
associated to the same request, since has already contributed to the
flooding. Note also that each of these packets has a max\_segment\_hops
field to prevent packets from travelling undefinitively.  

setup of the starting-segment: when this STARTING\_SEGMENT\_REQUEST packet reaches
the starting node (from a different link, of course), the starting
segment is found. Then the starting node sends back a SEGMENT\_CONFIRM
packet with the same id along the link from which it received the
STARTING\_SEGMENT\_REQUEST. Each node do same by changing its own status
from tvisited to visited. So the confirmation packet is sent back from
node to node and the starting segment is created. 

Note: A node knows to be tvisited due a  STARTING\_SEGMENT\_REQUEST.
Then, when receiving a SEGMENT\_REQUEST for other (not the starting)
segment, simply knows to cancel the tvisited status of the node and
its links, since if a segment arrived it means that a starting segment
has already been found a there’s no need to keep the tvisited status
for those broadcast status. What happens to nodes not receiving these
segment request? they simply remain tvisited and after a timeout reset
their state as free. But they could also remain tvisited for the
starting segment, since this doesn’t affect their behaviour for future
segment requests.  

failure: [TODO] should we handle this case ? i.e.
marking as terminal ?

\subsection{Finding other segments}

Injecting request: A node in the ASSIGNED state can initiate a search for a
segment, by sending a SEGMENT\_REQUEST with an across one of its free
links. Note that, since this is not the starting segment, in this case
the packet should not reach the initiator, but just another visited
node

setup of the segment: The find process is successfull when a visited
node receives the SEGMENT\_REQUEST packet. Then, a SEGMENT\_CONFIRM is
sent back along the path that originated the request.  Failing while
searching a segment: a node received a SEGMENT\_REQUEST packet but
matched one of the following conditions: the node is free but has no
more suitable free links (can’t forward the SEGMENT\_REQUEST) the node
is tvisited candidate for another find process the request packet
exceeded the max\_segment\_hops limit.  In all these cases the node
sends back a SEGMENT\_CANCEL along the proper link modifying its state
to free. Each of the subsequent nodes will forward this SEGMENT\_CANCEL
only if no more free links can be explored starting from them.


\subsection{Intra-node vs Inter-node parallelism}
In the processes described above, we assumed that each node in the
READY\_SEARCHING state can start a find segment process by injecting a
SEGMENT\_REQUEST. However, a more accurate decision when defining DiSR
should be made in order to decide whether or not the node must
actually perform this action. 
The point above is is strictly related to the general question: which
level of “parallelism” should we allow in DiSR ? The adopted approach
is that, although the nature of DiSR itself is intrinsicly parallel,
the use of parallelism makes things work in a more complex way. In
other words, DiSR is parallel when needed, but it does not exploit
parallelism as an “improving feature”. When not needed, things should
be serialized. For example: when a node is “ready searching” could
start several find segment process associated to the same segment, one
for each free link. But serializing this by investigating the free
links in order could be a simplest solution. We refer to this saying
that we avoid intra-node parallelism.  Note that while intra-node
parallelism can be avoided, trying to avoid inter-node parallelism
could be more complex than allowing the parallelism itself. Imagine
for example the effort of trying to coordinate nodes so that an unique
finding node process is actually running in the whole subnet. Thus, in
contrast to the intra-node parallelism, the inter-node parallelism is
a structural property of the DiSR algorithm and should not be avoided.
%------------------------------------------------------------------------------

\section{Detailed Node behaviour Model}

%TODO: add state machines

% TODO: fix, move elsewhere or delete
Assumption: when discussing of visited/tvisited nodes, we are assuming
with an id corresponding to the same subnet. Nodes involved in other
subnet processes can simply ignore these packets. (TBD: formalize more
clearly in which cases inter-subnet packets must be ignored, probably
this happens most of the times, but in some cases, e.g. when a
terminal node is found, some kind of communication packets are needed
in order to crearte the bridge between two subnets.) 

\subsection{Receiving a STARTING\_SEGMENT\_REQUEST}

When a node receives a STARTING\_SEGMENT\_REQUEST, with id = X,  there
are the following cases:

The node is visited with the same segID = x, that is it was the
initiator of the request (set visited by definition): a first segment
was found and should be confirmed sending back a SEGMENT\_CONFIRM
packet.  

The node is tvisited, with the same segID of the packet: due
the flooding mechanism, another starting segment packet reached a node
which was previously marked as tvisited from the same starting segment
search process. Since the node is tvisited with the same id, it means
that it already accomplished in the task of propagating that kind
packets, thus can simply ignore the event thrashing the packet.  If
the node it's not visited/tvisited: it should set itself as tvisited
and forward the packet along its free links, using flooding. TBD: how
the status should be changed ? depend on the flooding mechanism. Note
that these links don’t include the link from which the request was
received.  

The node it's visited, with different id in the packet.
CHECK:  it is not the initiator of the starting segment request, but
is already assigned to some segment!? assuming again we are discussing
of nodes belonging to the same subnet, this should NOT happen, since
the only visited node of the subnet should be the initial node from
which the STARTING\_SEGMENT\_REQUEST originated. Note again that we are
assuming that if the the node is visited or tvisited with the id of
another subnet can simply ignore the request.  The node it's tvisited
but it's not the initiator of the request (different id): same as
above, if we are assuming nodes of the same subnet, this shouldn’t
happen. If the node is of another subnet, can simply ignore the event.

\subsection{Receiving a SEGMENT\_REQUEST}
When a node receives a SEGMENT\_REQUEST, the are the following cases:

The node is visited (and belongs to the same subnet): a segment was
found and it should be confirmed sending back a SEGMENT\_CONFIRM
packet.  

The node is tvisited: it should discard the packet sending
back a SEGMENT\_CANCEL 

The node it's not visited/tvisited and has free
links: it marks itself as tvisited and forwards the SEGMENT\_REQUEST
(one request at time) to each of its free links, according to some
ordering

CHECK: the difference between confirming a STARTING\_SEGMENT\_REQUEST
and confirming a SEGMENT\_REQUEST is that in the first case the node
itself it's included in the segment , while in the second case the
receiving node DOES NOT belong to the segment found. The motivation is
that in the second case the node was already marked as visited because
of being assigned to some segment found before, while in the
starting-segment case, by definition, the first segment of a subnet
begins with the starting node, which is marked as visited by the SR
algorithm when searching for the first (starting) segment. Remember
that the first segment of a subnet is a loop beginning with the
starting node. 

\subsection{Receiving a SEGMENT\_CANCEL}
When a node receives a SEGMENT\_CANCEL packet it means that searching
for a segment along that path was unsuccessful. But if the node still
has other free links to try, it should forward a SEGMENT\_REQUEST to
the next link (next according to some internal ordering). So a node
forwards back the SEGMENT\_CANCEL packet along the link that originated
the SEGMENT\_REQUEST packet ONLY when there are no more free links to
try. If this is the case, the node modifies its status from tvisited
to free and fordwards the SEGMENT\_CANCEL packet to the link from which
the request was received. The process stops when the SEGMENT\_CANCEL
packet reach the starting node that originated the request.

\section{Nanoxim and simulation environment}

The simulator Nanoxim is a SystemC tool based on a minimal version of
the Noxim Network-on-Chip simulator. It's not a "faster noxim", but a
drastically reduced noxim the will be able to simulate a random
network on nodes. Current differences from noxim:

removed all routing algorithms that depend on a topology (only XY left
to test meshes) removed flit management, nodes only exchange packets
(e.g. no wormhole)

Future features:

topology agnostic able to test different nano-oriented communication
mechanism

%------------------------------------------------------------------------------

\section{Experiments}
% Absolute results:
% Comparison results:


%------------------------------------------------------------------------------

\section{Conclusions}
...

%------------------------------------------------------------------------------

\balance

\bibliographystyle{IEEEtran} 
\bibliography{bibliography}

%------------------------------------------------------------------------------

\end{document}
