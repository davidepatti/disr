\documentclass[conference]{IEEEtran}
\IEEEoverridecommandlockouts

%------------------------------------------------------------------------------

\usepackage{cite}
\usepackage{amsmath}
\usepackage{amsfonts}
\usepackage{amssymb}
\usepackage{graphicx}
\usepackage{url}
\usepackage{cite}
\usepackage{balance}
\usepackage{float}
\usepackage{threeparttable}

\usepackage{mathptmx}
\usepackage[scaled=.90]{helvet}
\usepackage{courier}

\usepackage{listings}
\usepackage{listings}
\lstset{
   language=C,
   basicstyle=\small,
   keywordstyle=\bfseries,
   identifierstyle=\ttfamily,
   stringstyle=\ttfamily,
   numbers=left,
   numberstyle=\tiny,
   stepnumber=1,
   numbersep=-5pt,
   showstringspaces=false
%   frame=single %trbl%
}

\usepackage[normalem]{ulem}

%------------------------------------------------------------------------------

\newcommand{\etal}{\emph{et al.}}
\newcommand{\eg}{\emph{e.g.}}
\newcommand{\ie}{\emph{i.e.}}
\newcommand{\etc}{\emph{etc.}}

%------------------------------------------------------------------------------

\begin{document}

%------------------------------------------------------------------------------

\title{A distribuited segment-based routing for bla bla...} 

\author{\IEEEauthorblockN{Maurizio Palesi} \IEEEauthorblockA{Affiliation1\\ email1} \and
  \IEEEauthorblockN{Davide Patti} \IEEEauthorblockA{Affiliation2\\ email2} \and
  \IEEEauthorblockN{Christof Teuscher} \IEEEauthorblockA{Affiliation3\\ email3}}

\maketitle

%------------------------------------------------------------------------------

\begin{abstract}
  ...
\end{abstract}

%------------------------------------------------------------------------------

\IEEEkeywords{...}

%------------------------------------------------------------------------------

\section{Introduction}
% intro to the potential of Self-assembled nano architectures
% irregular topology -> how to route ?


%------------------------------------------------------------------------------

\section{Related Work}
% - Up/Down etc..
% - SR is ok (topology independent) but requires the knowledge of the
% topology graph -> impossible in DNA self assembly
% - Our contribution: a distribuited approach for a segment-based
%   routing. Not the "optimal" SR, just a SR "that works" in such
%   complex irregular topologies where a graph could never be
%   available.
...

%------------------------------------------------------------------------------

\section{Proposal}
% - Draft of the DiSR idea
% - how it should work
...
%------------------------------------------------------------------------------

\section{DiSR execution model}
% - what is executed at each node, which information is required
...

%------------------------------------------------------------------------------

\section{Nanoxim and simulation environment}
% - Architecture of the nanoxim simulator
% - how simulations were conducted
...

%------------------------------------------------------------------------------

\section{Experiments}
% - Graphs, tables
% - comparison against other approaches
...

%------------------------------------------------------------------------------

\section{Conclusions}
...

%------------------------------------------------------------------------------

\balance

\bibliographystyle{IEEEtran} 
\bibliography{bibliography}

%------------------------------------------------------------------------------

\end{document}
