\documentclass[conference]{IEEEtran}
\IEEEoverridecommandlockouts

%------------------------------------------------------------------------------

\usepackage{cite}
\usepackage{amsmath}
\usepackage{amsfonts}
\usepackage{amssymb}
\usepackage{graphicx}
\usepackage{url}
\usepackage{cite}
\usepackage{balance}
\usepackage{float}
\usepackage{threeparttable}

\usepackage{mathptmx}
\usepackage[scaled=.90]{helvet}
\usepackage{courier}

\usepackage{listings}
\usepackage{listings}
\lstset{
   language=C,
   basicstyle=\small,
   keywordstyle=\bfseries,
   identifierstyle=\ttfamily,
   stringstyle=\ttfamily,
   numbers=left,
   numberstyle=\tiny,
   stepnumber=1,
   numbersep=-5pt,
   showstringspaces=false
%   frame=single %trbl%
}

\usepackage[normalem]{ulem}

%------------------------------------------------------------------------------

\newcommand{\etal}{\emph{et al.}}
\newcommand{\eg}{\emph{e.g.}}
\newcommand{\ie}{\emph{i.e.}}
\newcommand{\etc}{\emph{etc.}}

%------------------------------------------------------------------------------

\begin{document}

%------------------------------------------------------------------------------

\title{A Distribuited Segment-based Approach for Topology-agnostic Nano Networks Routing} 

\author{
  \IEEEauthorblockN{Vincenzo Catania} \IEEEauthorblockA{Affiliation3\\ email3}} \and
  \IEEEauthorblockN{Davide Patti} \IEEEauthorblockA{Affiliation2\\ email2} 

\maketitle

%------------------------------------------------------------------------------

\begin{abstract}
In this paper we present a first effort for a distributed
segment-based approach to routing which exploits the already-proven
qualities of segment-based routing without requiring as graph
topology as input, and also without requiring a centralized setup
phase.
\end{abstract}

%------------------------------------------------------------------------------

\IEEEkeywords{...}

%------------------------------------------------------------------------------

\section{Introduction}
Exploring long term alternatives to the CMOS technology is gaining
more and more revelance as the scaling trend of such devices continue
to introduce new challenges. Power density, defect tolerance, testing
costs and many other issues are only partially mitigated by multicore
approaches~\ref{TODO}, but growing computing requirements will eventually need
even more radical architectural modifications and new paradigms in
future computer design.

DNA-self assembled nanoscale architectures~\ref{TOOD} are emerging as promising
technology due its tera/peta scale of integration, defect tolerance
tolerance and potential computing capabilities~\ref{TODO}. This
technologies is certainly still at its early stage of development,
however different laboratory demos and prof-of-concepts architectures
have been presented~\ref{TODO}.

The main idea beyond this approach is to exploit the regularity and
stabiliy of DNA sequences in orders to create grids onto which
nano-devices (e.g. nanowires and CNFET~\ref{TODO}) can be attached by
designing approapriate DNA tags. A detailed description of the
chemical properties involved is beyond the scope of this paper, since
we focus of the main challenges that this new fabrication process
introduces in Computer Design. These are:
\begin{itemize}
\item small scale control of placement and connectivity
\item large scal randomness
\item high defect rates
\end{itemize}


This aspects of a DNA-self assembled process leads some architectural implications
when approaching to the Computer Design. Architecture must consist of
a distributed network of small computing and storage nodes, randomly
connected. Data will be travelling on packets that must be routed
using a topology-agnostic strategy, since no particular regularity can
be assumed in such network. 

In this paper we present a first effort for a distributed
segment-based approach to routing which exploits the already-proven
qualities of segment-based routing without requiring as graph
topology as input, and also without requiring a centralized algorithm
which perform a \"segment setup\" phase.

% irregular topology -> how to route ?


%------------------------------------------------------------------------------

\section{Related Work}
% - Up/Down etc..
% - SR is ok (topology independent) but requires the knowledge of the
% topology graph -> impossible in DNA self assembly
% - Our contribution: a distribuited approach for a segment-based
%   routing. Not the "optimal" SR, just a SR "that works" in such
%   complex irregular topologies where a graph could never be
%   available.
...
Different architectures using nano-scale devices have been presented.A
general purpose architecture for Self-Assembled Nano-Electronics has
been presented to demostrate the potential of
DNA-selfassembly~\ref{TODO}

%------------------------------------------------------------------------------

\section{Proposal}
% - Draft of the DiSR idea
% - how it should work
...
%------------------------------------------------------------------------------

\section{DiSR execution model}
% - what is executed at each node, which information is required
...

%------------------------------------------------------------------------------

\section{Nanoxim and simulation environment}
% - Architecture of the nanoxim simulator
% - how simulations were conducted
...

%------------------------------------------------------------------------------

\section{Experiments}
% - Graphs, tables
% - comparison against other approaches
...

%------------------------------------------------------------------------------

\section{Conclusions}
...

%------------------------------------------------------------------------------

\balance

\bibliographystyle{IEEEtran} 
\bibliography{bibliography}

%------------------------------------------------------------------------------

\end{document}
