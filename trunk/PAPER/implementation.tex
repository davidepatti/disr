
\section{Implementation Issues}
\label{sec:implementation}
Althought a completed and detailed floorplan of the node is not the
focus of this work, in this section we want to give a basic idea of
which are the main elements required to support this kind of
architecture and the DiSR approach. There are three main classes of
node elements:
\begin{itemize}
\item Node-specific: components (such ALUs, memories) that are
strictly related to the node functionality and role inside a given
networks: e.g. is that a computation or storage node ?
\item Node communication:  these are elements (such as transceivers,
buffers) required to the node to communicate with its neighbors,
indepentently from node functionality and DiSR implementation
\item DiSR-specific: all the hardware, such as control logic and
configuration, required to implement the proposed DiSR approach.
\end{itemize}

\begin{figure}
  \centering
  \includegraphics[width=0.50\textwidth]{pictures/router_schematic.eps}
  \caption{Schematic of the router. Top-level view (top). Block
  implementing the routing algorithm (middle). Block implementing the
  TODO
  }
  \label{fig:schematic}
\end{figure}

Figure~\ref{fig:schematic} shows the schematic of a router for
a mesh based network topology implementing DiSR . As shown, the top level scheme of the router consists of a
set of ...

The content of a routing algorithm block is more accurately shown in
the central part of Figure~\ref{fig:schematic}. Each routing algorithm
block consists of ...

Let us now look inside the \emph{DiSR} block. A formal description of
DiSRalgorithm has already been given in Figure~\ref{alg:disr}, thus we
will focus on the architectural elements to be implemented.
	
